\documentclass[a4paper,10pt]{memoir}

\usepackage[utf8]{inputenc}
\usepackage[T1]{fontenc}
\usepackage[margin=3cm]{geometry}
\usepackage[german]{babel}
\usepackage{graphicx}
\usepackage{longtable}
\usepackage{pbox}
\usepackage{datetime}
\usepackage[usenames,dvipsnames]{color}
\usepackage{multicol}
\usepackage[tracking=true]{microtype}

\setlength{\parindent}{0pt}
\setlength\extrarowheight{.5em}

\pagestyle{empty}

\begin{document}

\section*{\color{MidnightBlue} Curriculum Vitae}

\begin{flushright}
  Gültig \monthname\,\the\year
\end{flushright}

\input{personal_data_german}
\\
\\

\subsection*{Studium}
\vspace*{-\baselineskip}
% newline in table acc. to http://tex.stackexchange.com/a/2442/107834
\begin{longtable}{@{}p{.2\textwidth} p{.8\textwidth}}
  09/2017 &
  Ferienakademie der TU München, Südtirol \newline
  Kurs: Using machine learning and simulations to create animations
  \\
  ab 10/2016 &
  Technische Universität München  \newline
  Informatik Aufbaustudium
  \\
  09/2016 &
  Summer School on Medical Imaging, ETH Zürich
  \\
  04/2014 - 06/2016 &
  Technische Universität München \newline
  Physik (Physik der kondensierten Materie), M.Sc., Abschlussnote: 1.48
  \\
  04/2015 - 04/2016 &
  Technische Universität München, Lehrstuhl E17 (Biomedizinische Physik) \newline
  Masterarbeit, Note: 2.0
  \\
  08/2013 - 12/2013 &
  Universitet Uppsala, Schweden \newline
  Department of Physics and Astronomy, Erasmus SMS Auslandsaufenthalt
  \\
  10/2011 - 07/2013 &
  Technische Universität München \newline
  Physik (Physik der kondensierten Materie), B.Sc., Abschlussnote: 1.5
  \\
  04/2013 - 07/2013 &
  \SetTracking{encoding=*}{-10}\lsstyle
  Technische Universität München, Lehrstuhl E20 (Molekulare Nanowissenschaften) \newline
  \SetTracking{encoding=*}{0}\lsstyle
  Bachelorarbeit, Note: 1.3
  \\
  07/2011 - 09/2011 &
  Technische Universität München \newline
  twoinone-Studium Physik
  \\
  10/2010 - 03/2011 &
  Technische Universität Kaiserslautern \newline
  Fernstudium Physik
\end{longtable}

\subsection*{Schule}
\vspace*{-\baselineskip}
\begin{longtable}{@{}p{.2\textwidth} p{.8\textwidth}}
  09/2001 - 06/2010 &
  Dom-Gymnasium Freising  \newline
  Allgemeine Hochschulreife, Note: 1.0
\end{longtable}

\subsection*{Zivildienst}
\vspace*{-\baselineskip}
\begin{longtable}{@{}p{.2\textwidth} p{.8\textwidth}}
  01/2011 - 06/2011 &
  Technische Universität München, Weihenstephan \newline
  Lehrstuhl für Renaturierungsökologie
\end{longtable}

\subsection*{Berufserfahrung}
\vspace*{-\baselineskip}
\begin{longtable}{@{}p{.2\textwidth} p{.8\textwidth}}
  ab 08/2017 &
  ReactiveRobotics GmbH, München (Medizinrobotik für Intensivstationen) \newline
  Software-Entwickler (GUI-Entwicklung, Robotik-Steuerung)
  \\
  09/2016 - 08/2017 &
  PreciBake GmbH, München (industrielle Ofensteuerungen) \newline
  Software-Entwickler (Bildverarbeitung, ROS, eingebettete Systeme)
  \\
  04/2015 - 06/2016 &
  ImFusion GmbH, München (Software für medizinische Bildverarbeitung) \newline
  Masterarbeit (Rauschfilterung in Computertomographie)
  \\
  12/2014 - 02/2015, \newline
  06/2014 - 08/2014 &
  Technische Universität München, Lehrstuhl E17 (Biomedizinische Physik) \newline
  Studentische Hilfskraft (GUI-Entwicklung)
  \\
  01/2014 - 03/2014 &
  TNG Technology Consulting, Unterföhring b. München \newline
  Praktikant (internes Projekt)
\end{longtable}

\subsection*{Programmiersprachen}
\begin{itemize}
  \item Python (numpy, scipy, sk-learn, sk-image, unittest, matplotlib)
  \item C/C++ (Eigen, fftw)
  \item Libraries/Frameworks wie z.B. Qt, OpenCV, ROS
  \item Java
\end{itemize}

\subsection*{Sonstige Software- und Betriebssystemkenntnisse}
\vspace*{-\baselineskip}
\begin{multicols}{2}
\begin{itemize}
  \item GPGPU (OpenCL, Cuda)
  \item Computergraphik (OpenGL)
  \item VCS (git, svn)
  \item \LaTeX
  \item Linux, Vim, Bash
  \item Open-Source-Projekte auf github.com/pylipp
\end{itemize}
\end{multicols}

\subsection*{Sprachen}
\vspace*{-\baselineskip}
\begin{multicols}{3}
  \begin{itemize}
    \item Deutsch (Muttersprache)
    \item Englisch (fließend)
    \item Französisch (gut)
    \item Schwedisch (Grundk.)
    \item Spanisch (Grundk.)
  \end{itemize}
\end{multicols}

\subsection*{Vereine/Interessen}
\vspace*{-\baselineskip}
\begin{longtable}{@{}p{.2\textwidth} p{.8\textwidth}}
  2010 - heute &
  Studententheatergruppe ``KreativesSchauspielEnsemble'' \newline
  Gründungsmitglied und Kassenwart
  \\
  2004 - heute &
  Badminton Club Freising \newline
  Mannschaftsführer (seit 2009), Schriftführer (seit 2010)
\end{longtable}
\end{document}
